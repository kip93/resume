%%%%%%%%%%%%%%%%%%%%%%%%%%%%%%%%%%%%%%%%%%%%%%%%%%%%%%%%%%%%%%%%%%%%%%%%%%%%%%%%
% An example Résumé using a custom template.                                   %
%                                                                              %
% Author:   Kip (https://github.com/kip93/).                                   %
% Source:   https://github.com/kip93/resume/                                   %
% License:  BSD 3-Clause                                                       %
% Created:  2020-09-27                                                         %
% Updated:  2020-12-20                                                         %
%%%%%%%%%%%%%%%%%%%%%%%%%%%%%%%%%%%%%%%%%%%%%%%%%%%%%%%%%%%%%%%%%%%%%%%%%%%%%%%%

% This loads the `resume.cls` file which defines all the commands used beneath.
\documentclass{resume}

% Auto generator of placeholder text, can be removed (once all placehodlder text `\lipsum[...]` has been removed).
\RequirePackage{lipsum}

% All configuration should be made inside of this `resume` block.
\begin{resume}

    % Header
    % ------------------------------------------------------------------------ %

    % Path to a square profile picture.
    \profilepicture{./example.png}
    % Your name (Last name, then first name).
    \name{Doe}{John}
    % Your job title.
    \jobtitle{Professional Googler}
    % Your birthday (format {YYYY}{MM}{DD}).
    \birthday{1970}{01}{01}
    % Your contact e-mail.
    \email{john@doe.com}
    % Your phone number.
    \phone{+1 (123) 4567-890}
    % Your linkedin profile (the last part in the url, right after https://www.linkedin.com/in/).
    \linkedin{johndoe}  % Optional.
    % Your GitHub username.
    \github{johndoe70}  % Optional.
    % You can also GitLab username instead.
    % \gitlab{johndoe70}  % Optional.
    % Or even a BitBucket username.
    % \bitbucket{johndoe70}  % Optional.

    % Left column
    % ------------------------------------------------------------------------ %

    % Your summary / introduction text (`\lipsum[1-1]` is just placeholder text, and can be safely removed).
    \summary{\lipsum[1-1]}  % Optional.

    % A set of skills under a given category.
    \begin{skillset}{Markup}
        % A single skill that corresponds to the encompasing skill set.
        % The format is {Skill name}{Skill score in a 1-5 range}{Score tooltip (optional)}.
        \skill{Markdown}{5}{}
        % You can declare as many skills as you want inside a skill set.
        \skill{HTML}{4}{}
        % Skills names support expressions.
        \skill{\LaTeX}{5}{}
    \end{skillset}

    % You can also declare as many skill sets as you need, but there must be at least one.
    \begin{skillset}{Programming}
        % Skill sets should always have at least a single skill in them.
        \skill{Python}{5}{}
        \skill{Bash}{5}{}
        \skill{Java}{5}{}
        \skill{C/C++}{3}{}
    \end{skillset}

    \begin{skillset}{Operating Systems}
        % If you want to assert dominance with your skills, you can set them to
        % be over 9000, the resume will handle them (and just cast them to range).
        \skill{Linux}{9001}{}
        % You can even use negatives.
        \skill{Windows}{-1}{}
        % But not decimal numbers.
        % \skill{MacOS}{4.2}
    \end{skillset}

    \begin{skillset}{Languages}
        % You may want to override the score tooltip for non-techical skills.
        \skill{English}{5}{Native}
        \skill{Spanish}{4}{C1 certified}
        % You can create links to provide more context.
        \skill{\link{https://lotr.fandom.com/wiki/Quenya}{Quenya}}{2}{Learning as a hobby}
    \end{skillset}

    % And the different skill levels have default tooltips to better describe
    % the mastery (or lack thereof) of each one.
    \begin{skillset}{All levels}
        % 5: I only use Google because it is easier.
        \skill{Level 5}{5}{}
        % 4: I can do some stuff without Google.
        \skill{Level 4}{4}{}
        % 3: I know exactly what to Google.
        \skill{Level 3}{3}{}
        % 2: I understand what I am copying from Google.
        \skill{Level 2}{2}{}
        % 1: I know some keywords that help me Google.
        \skill{Level 1}{1}{}
        % 0: I utterly stumble my way copying from Google.
        \skill{Level 0}{0}{}
    \end{skillset}

    % Right column
    % ------------------------------------------------------------------------ %

    % Your work experience. Format is {Start}{End}{Eployer}{Job title}{Job description}
    % (`\lipsum[2-2]` is just placeholder text, and can be safely removed).
    \experience{2019/03}{Present}{Speedwagon Fundation}{Lead Googler}{\lipsum[2-2]}  % Optional.
    % You can declare as many or as few experience entries as needed.
    % The description field is optional and if empty it will be properly handled.
    \experience{2017/08}{2019/02}{Speedwagon Fundation}{Intern Googler}{}  % Optional.

    % Your education. Format is {Start}{End}{University}{Title}{Notes} (`\lipsum[3-3]` is just placeholder text, and can be safely removed).
    \education{2016}{2018}{University of Somewhere}{M.Sc. in Googling}{\lipsum[3-3]}  % Optional.
    % You can declare as many or as few education entries as needed.
    % The notes field is optional and if empty it will be properly handled.
    \education{2013}{2016}{University of Somewhere}{B.Sc. in Googling}{}  % Optional.

    % Your publications. Format is {Date}{Title}{URL}{Notes} (`\lipsum[4-4]` is just placeholder text, and can be safely removed).
    \publication{2020}{How to Google with the best of them}{https://www.example.com/}{\lipsum[4-4]}  % Optional.
    % You can declare as many or as few publications as needed.
    % The description field is optional and if empty it will be properly handled.
    \publication{2019}{Googling like a pro}{https://www.example.com/}{}  % Optional.

% This is what actually builds the Résumé, no other loose text should be placed.
\end{resume}
